\documentclass{beamer}
\usepackage[T1]{fontenc}
\usepackage[utf8]{inputenc}
\usepackage{ru}
\setbeamercovered{transparent}

\author{Steven Bronsveld \and Jelmer Firet}
\title[Scalar Actions]{Scalar Actions in Lean's Mathlib}
\subtitle{Based on a paper by Eric Wieser\cite{wieserScalarActionsLean2021}}

\institute[Radboud University Nijmegen]{Radboud University Nijmegen}

\usepackage{amsmath}
\usepackage{amssymb}
\usepackage{upgreek}
\usepackage{color}
\usepackage{listings}
\usepackage{lstautogobble}

\definecolor{keywordcolor}{rgb}{0.7, 0.1, 0.1}   % red
\definecolor{commentcolor}{rgb}{0.4, 0.4, 0.4}   % grey
\definecolor{symbolcolor}{rgb}{0.0, 0.1, 0.6}    % blue
\definecolor{sortcolor}{rgb}{0.1, 0.5, 0.1}      % green
\definecolor{errorcolor}{rgb}{1, 0, 0}           % bright red
\definecolor{stringcolor}{rgb}{0.5, 0.3, 0.2}    % brown

\def\lstlanguagefiles{lstlean.tex}
\lstset{language=lean, basicstyle=\footnotesize, numbers=left, autogobble=true}

\usepackage[style=numeric]{biblatex}
\addbibresource{bibliography.bib}

\usepackage{xparse}
\NewDocumentCommand{\rot}{O{60} O{1.3em} m}{\makebox[#2][l]{\rotatebox{#1}{#3}}}

\newcommand{\yes}{\checkmark}
\newcommand{\no}{}

\begin{document}
	\begin{frame}[t]
	\titlepage
	\end{frame}

	\section*{Structures} % (fold)
	\label{sec:structures}
	\begin{frame}
		\frametitle{Overview of structures with multiplication}
		\begin{tabular}{l*{14}{@{}c}}
			structure & 
				\rot[0]{0} &
				\rot[0]{1} &
				\rot{$a\cdot b$} &
				\rot{$a\cdot0=0\cdot a=0$} &
				\rot{$a\cdot 1=1\cdot a=a$} &
				\rot{$(a\cdot b)\cdot c=a\cdot(b\cdot c)$} &
				\rot{$a\cdot b = b \cdot a$} &
				\rot{$a+b$} &
				\rot{$0 + a = a + 0 = a$} &
				\rot{$(a+b)+c=a+(b+c)$} &
				\rot{$a + b = b + a$} &
				\rot{$(a+b)\cdot c=a\cdot c + b \cdot c$}&
				\rot{$a\cdot(b+c)=a\cdot b+a\cdot c$}\\
			\hline
			has\_mul			&\no	&\no	&\yes	&\no	&\no	&\no	&\no	&\no	&\no	&\no	&\no	&\no	&\no	\\
			mul\_zero\_class	&\yes	&\no	&\yes	&\yes	&\no	&\no	&\no	&\no	&\no	&\no	&\no	&\no	&\no	\\
			monoid				&\no	&\yes	&\yes	&\no	&\yes	&\yes	&\no	&\no	&\no	&\no	&\no	&\no	&\no	\\
			monoid\_with\_zero	&\yes	&\yes	&\yes	&\yes	&\yes	&\yes	&\no	&\no	&\no	&\no	&\no	&\no	&\no	\\
			semiring			&\yes	&\yes	&\yes	&\yes	&\yes	&\yes	&\no	&\yes	&\yes	&\yes	&\yes	&\yes	&\yes	\\
			comm\_semiring		&\yes	&\yes	&\yes	&\yes	&\yes	&\yes	&\yes	&\yes	&\yes	&\yes	&\yes	&\yes	&\yes	\\
		\end{tabular}
	\end{frame}
	
	\begin{frame}[t]
	\frametitle{Overview of structures with scalar multiplication}
		\begin{tabular}{l*{11}{@{}c}}
			structure &
				\rot[50][1.4em]{$r\bullet m$} &
				\rot[50][1.4em]{$r\bullet 0_M=0_M$} &
				\rot[50][1.4em]{$0_R\bullet m=0_M$} &
				\rot[50][1.4em]{$1_R\bullet m=m$} &
				\rot[50][1.4em]{$(r\cdot s)\bullet m=r\bullet (s\bullet m)$} &
				\rot[50][1.4em]{$r\bullet(m+n)=r\bullet m+r\bullet n$} &
				\rot[50][1.4em]{$(r+s)\bullet m=r\bullet m+s\bullet m$} &
				\rot[50][1.4em]{$r\bullet m=r_M\cdot m=m\cdot r_M$} &
				\rot[50][1.4em]{$r\bullet (m\cdot n)=(r\bullet m)\cdot(r\bullet n)$} &
				\rot[50][1.4em]{$r\bullet 1_M=1_M$} &
				\\
			\hline
			has\_scalar				&\yes	&\no	&\no	&\no	&\no	&\no	&\no	&\no	&\no	&\no	&\\
			smul\_with\_zero		&\yes	&\yes	&\yes	&\no	&\no	&\no	&\no	&\no	&\no	&\no	&\\
			mul\_action				&\yes	&\no	&\no	&\yes	&\yes	&\no	&\no	&\no	&\no	&\no	&\\
			mul\_action\_with\_zero	&\yes	&\yes	&\yes	&\yes	&\yes	&\no	&\no	&\no	&\no	&\no	&\\
			distrib\_mul\_action	&\yes	&\yes	&\no	&\yes	&\yes	&\yes	&\no	&\no	&\no	&\no	&\\
			module					&\yes	&\yes	&\yes	&\yes	&\yes	&\yes	&\yes	&\no	&\no	&\no	&\\
			algebra					&\yes	&\yes	&\yes	&\yes	&\yes	&\yes	&\yes	&\yes	&\no	&\no	&\\
			mul\_semiring\_action	&\yes	&\yes	&\no	&\yes	&\yes	&\yes	&\no	&\no	&\yes	&\yes	&\\
		\end{tabular}
	\end{frame}

	\begin{frame}[t, fragile]
	\frametitle{has\_scalar}
		\begin{block}{Structure (\texttt{algebra.group.defs})}
			\begin{lstlisting}
			class has_scalar (M : Type*) (α : Type*) :=
			(smul : M → α → α)
			infixr ` • `:73 := has_scalar.smul
			\end{lstlisting}
		\end{block}

		\begin{block}{Instance (\texttt{group\_theory.group\_action.defs})}
			\begin{onlyenv}<1>
			\begin{lstlisting}
				instance has_mul.to_has_scalar (α : Type*) [has_mul α]
				: has_scalar α α := ⟨(*)⟩
			\end{lstlisting}
			\end{onlyenv}

			\begin{onlyenv}<2>
			\begin{lstlisting}
				instance has_mul.to_has_scalar (α : Type*) [has_mul α]
				: has_scalar α α := { smul := mul }
			\end{lstlisting}
			\end{onlyenv}
		\end{block}
	\end{frame}

	\begin{frame}[t, fragile]
	\frametitle{function.has\_scalar}
		\begin{block}{Instance}
			\begin{lstlisting}
				instance function.has_scalar (I α : Type*) [has_mul α]
				: has_scalar α (I → α) := { smul := λ r v, (λ i, r * v i)}
			\end{lstlisting}
		\end{block}
		\pause
		What about \lstinline{has_scalar M (I → κ → α)}?
		\pause
		\begin{block}{Instance}
			\begin{lstlisting}
				instance function.has_scalar (I M α : Type*) [has_scalar M α]
				: has_scalar M (I → α) := { smul := λ r v, (λ i, r • v i) }
			\end{lstlisting}
		\end{block}
		\pause
		What about \lstinline{has_scalar M (Π i : I, f i)}?
		\pause
		\begin{block}{Instance (\texttt{group\_theory.group\_action.pi)}}
			\begin{lstlisting}
				instance pi.has_scalar (...) [Π i, has_scalar M (f i)]
				: has_scalar M (Π i : I, f i) := { smul := λ r v, (λ i, r • v i) }
			\end{lstlisting}
		\end{block}
	\end{frame}
	% section structures (end)

	\section{}
	\renewcommand{\section}[2]{}
	\begin{frame}[t]
	\frametitle{Bibliography}
		\printbibliography
	\end{frame}
\end{document}
